%%%%% --------------------------------------------------------------------------------
%%
%%%%******************************* Main Content *************************************
%%
%%% ++++++++++++++++++++++++++++++++++++++++++++++++++++++++++++++++++++++++++++++++++




\section{初步方案}

\subsection{总体方案}
\sectionnotes{(阐述本项目研究的总体思路、技术路线。)}



\subsection{主要技术分析}
\sectionnotes{(分析识别实现研究目标和技术方案的主要技术;针对每一项技术,论述拟采取的技术途径、解决办法。)
	)}

\subsection{研制周期}

\subsubsection{项目周期}
\sectionnotes{(项目研究的自然年数。)}

2年。

\subsubsection{进度安排}
\sectionnotes{(针对研究内容,逐项分解并明确具体进度安排,精确到月份。)}

2019.1-2019.5:项目总体设计。包括资料收集与分析,不同类型数据集的收集,实验环境的准备。形成项目的总体设计报告和详细的研究计划。

2019.6-2020.7:关键技术攻关。包括迁移学习算法和图卷积网络的研究。在国内外权威刊物发表学术论文,其中SCI论文 2篇,EI论文 2 篇。申请国家发明专利2-4项。

2020.8-2020.10: 进一步扩展。在之前研究的基础上,将图像语义分割扩展到实例级别上。在国内外权威刊物等发表学术论文,其中SCI 论文1篇,EI 论文2 篇。申请国家发明专利2-4项。

2020.11-2020.12:技术总结,文档撰写,项目结题验收。


\subsection{项目组组成}
\subsubsection{项目组负责人}
\sectionnotes{(简要介绍其相关领域科研经历,承担的重大科研任务,及获奖、专家情况。)}

\subsubsection{项目组}
\sectionnotes{(列表形式,简要描述项目组人员的学历、目前所从事的专业领域、在本项目中所承担的工作任务等。)}



%%% ++++++++++++++++++++++++++++++++++++++++++++++++++++++++++++++++++++++++++++++++++
