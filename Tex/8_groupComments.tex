%%%%% --------------------------------------------------------------------------------
%%
%%%%******************************* Main Content *************************************
%%
%%% ++++++++++++++++++++++++++++++++++++++++++++++++++++++++++++++++++++++++++++++++++


\section{开题报告评议小组意见及评议结果}
\begin{mdframed}[everyline=true]

\begin{enumerate}[label={(\arabic*)},labelsep= 3 pt]
	\item {\songti 选题依据、研究内容、研究方案及技术路线的科学性、可行性及创新性的评价}
	
	\quad\quad  谭XX同学的硕士学位论文深入分析国内外的理论观点和技术方案,对监控视频异常事 件检测思路比较清晰,研究方法具有创新性,选题具有重要的应用价值。论文研究内容与工 作量适合硕士学位论文的要求,论文研究方法可行。
	经评议小组讨论,一致同意谭XX同学的硕士学位论文开题报告。
	
	\item  {\songti 存在的主要问题和修改建议 }
	\begin{enumerate}[label={\arabic*)},labelsep=3 pt]
		\item 论文不必提出“广义异常事件”的概念,只需说明工作内容为异常事件检测而不是区分 异常事件的具体类别。
		\item 异常事件检测如果不限定应用场景范围难度和工作量较大,建议先设定一个具体的场景 以降低难度,逐步推进。
		\item 光流的导数(加速度)也可以作为特征向量的一个维度。
		\item 如果采用的方法是无监督的可能更具有实用性。
	\end{enumerate}

	\item  {\songti 开题报告评议结果}
	
    $\square$  {\songti 通过} \quad\quad \quad \quad  $\square$ {\songti 不通过,且要求在2个月内重新组织开题}
	\\[20pt]
	
	
	{  \songti
	\quad\quad\quad\quad \quad\quad\quad\quad 	\quad\quad\quad\quad \quad\quad\quad\quad 组长(签名):
	
	\quad\quad\quad\quad \quad\quad\quad\quad 	\quad\quad\quad\quad \quad\quad\quad\quad \quad\quad\quad 年  \quad\quad 月  \quad\quad 日
	\\
    }
\end{enumerate}
\end{mdframed}
\vspace{-4pt}
{
	\noindent
\begin{tabular*}{0.999\textwidth}{| c  | c | c | p{0.5\textwidth}< {\centering} | p{0.157\textwidth}<{\centering}|}
%	\hline 
    \multicolumn{5}{|c|}{	\songti 开题报告评议小组组成}	\\
	\hline
	{\songti 组成} & {\songti 姓名} & {\songti 职称} &  {\songti  所在单位} & {\songti 本人签名}  \\
	\hline 
	{\songti 组长}     &  张老师 & 教授 &  五院XXXX研究所 &  \\ 
	\hline 
	\multirow{4}{8pt}{\songti 成员}  & 李老师  &  教授&  五院XXXX研究所 & \\ 
	\cline{2-5}
	   &  王老师 &  教授&  X院XXXX研究所 & \\ 
	\cline {2-5}
	   &  谭老师 &  教授&  五院XXXX研究所 & \\ 
	\cline {2-5}
	   &  老老师 &  教授&  五院XXXXXX系 & \\ 
	\hline 
	{\songti 秘书}   & 赖老师 & 讲师 &  五院XXXXXX系 & \\ 
	\hline 
\end{tabular*} 
   \indent
}


%%% ++++++++++++++++++++++++++++++++++++++++++++++++++++++++++++++++++++++++++++++++++
