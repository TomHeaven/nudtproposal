%%%%% --------------------------------------------------------------------------------
%%
%%                           Document Template of NUDT proposal
%%
%%%%% --------------------------------------------------------------------------------
%% Copyright (C) Hanlin Tan <hanlin_tan@nudt.edu.cn> 
%% This is free software: you can redistribute it and/or modify it
%% under the terms of the GNU General Public License as published by
%% the Free Software Foundation, either version 3 of the License, or
%% (at your option) any later version.
%%%%% --------------------------------------------------------------------------------
%% Last Updated: 2017.01.06
%%%%************************ Document Class Declaration ******************************
%%
%\documentclass{ctexart}

\documentclass{Style/nudtproposal}% thesis template of UCAS
%% Multiple optional arguments:
%% [scheme = plain] % for thesis writing of international students
%% [<singlesided|doublesided|printcopy>] % single-sided, double-sided, or print layout
%% [draftversion] % show draft version information, default is no show
%% [fontset = <adobe|...>] % specify font set, default is automatic detection
%% [standard options for ctex class]
%%%%% --------------------------------------------------------------------------------
%%
%%%%************************* Command Define and Settings ****************************
%%
\usepackage{Style/commons}% common settings
%% usage: \usepackage[option1,option2,...,optionN]{commons}
%% Multiple optional arguments:
%% [myhdr] % one available header and footer style, will enable fancyhdr
%% [lscape] % provide landscape layout environment
%% [geometry] % configure page layout by geometry package
%% [list] % enable enhanced list environments, useful for Algorithm and Coding
%% [color] % enable color package to use color, default package is xcolor
%% [background] % enable page background, will auto enable color package
%% [tikz] % enable tikz for complex diagrams, will auto enbale color package
%% [table] % enable a table package for complex tables, default is ctable
%% [math] % enable some extra math packages
\usepackage{Style/custom}% user defined commands
\usepackage[backend=biber, bibstyle=gb7714-2015,%nature,%%加载biblatex宏包,使用参考文献
citestyle=gb7714-2015,%,backref=true%%其中后端backend使用biber
gbnamefmt=lowercase,
gbpub=false % 是否显示“未知出版商”等信息从而更加符合gb7714-2015样式
]{biblatex}%标注(引用)样式citestyle,著录样式bibstyle都采用gb7714-2015样式
% 设置参考文献文件
\addbibresource{Biblio/ref.bib}
\usepackage{Style/nudtstyle} % 包含作者自定义的格式和命令
\enabletablebib{yes}   % 参考文献是否放入表格,默认yes。

%%%%% ---------------------------------------------------------------------------------
%%%%% ---------------警告:以上内容请勿随意修改,除非你清楚自己在做什么------------


%%%%% --------------提示:修改本节内容用于设置文档,请仔细阅读---------------------
%% 
%% 编译环境:texlive-2015或者texlive-2019。
%% 推荐IDE:texstudio。
%% 编译选项:tex编译器选择xelatex, 参考文献编译器选择biber(不能用bibtex)!
%% 以上环境配置经过作者测试,确定可以正常使用。

%% 以下参数用于设置文档首页和页眉信息
\proposaltype{doctor}          % 研究生类别:硕士设置为master,博士设置为doctor 
\enabletableofcontents{no}   % 是否生成目录:如果需要目录设置为yes,否则设置为no。我校开题报告默认没有目录
\proposalnumber{\underline{\hbox to 10mm{}}}          % 编号:默认是下划线,如果你知道编号,设为真实编号
\classification{公开}              % 密级:公开,秘密,机密或者绝密
\nudttitle{国防科学技术大学开题报告}{\LaTeX{} 模板} % 因title一般都很长需要两行,第一参数为第一行内容,第二个参数为第二行内容
\author{谭同学}                    % 作者
\authorid{160590xx}            % 学号
\advisor{张老师}                   % 导师
\advisortitle{教~~~~授}         % 职称
\degreetype{工学}                 % 学位类别
\major{控制科学与工程}          % 一级学科
\field{图像处理}                      % 研究方向
\institute{信息系统与管理学院}% 学院
\chinesedate{2017~年~03~月~01日} % 开题日期
\formdate{二〇一八年一月}     % 制表月份

%% 在设置完以上参数后,修改Tex文件下对应文件以完成开题报告。
%%%%% ---------------------------------------------------------------


%%%%% ---------------警告:以下内容请勿随意修改,除非你清楚自己在做什么------------

%%%%******************************** Content *****************************************
%%
\begin{document}
%%
%%%%% --------------------------------------------------------------------------------
%%
%%%%******************************** Frontmatter *************************************
%%
\pagenumbering{roman}% restart page numbers with arabic style
%%% Generate Title
%%
\maketitle

%%%%% --------------------------------------------------------------------------------
%%
%%%%******************************** Mainmatter **************************************
%%

%% 添加正文内容
\pagenumbering{arabic}% restart page numbers with arabic style
\mdfsetup{skipabove=0pt,skipbelow=0pt}
%% 包含正文各个章节,请编辑章节文件修改相应的内容
% 正文字号
\zihao{5}
\input{Tex/1_background}%   \include ?= \input + \clearpage
\clearpage
%%%%% --------------------------------------------------------------------------------
%%
%%%%******************************* Main Content *************************************
%%
%%% ++++++++++++++++++++++++++++++++++++++++++++++++++++++++++++++++++++++++++++++++++




\section{文献综述}
\begin{mdframed}[everyline=true]
	
   很多文献\upcite{yu2018spider}……  
  \\[20 cm]
\end{mdframed}



%%% ++++++++++++++++++++++++++++++++++++++++++++++++++++++++++++++++++++++++++++++++++
%
\clearpage
%%%%% --------------------------------------------------------------------------------
%%
%%%%******************************* Main Content *************************************
%%
%%% ++++++++++++++++++++++++++++++++++++++++++++++++++++++++++++++++++++++++++++++++++




\section{研究内容}
\begin{mdframed}[everyline=true]

\subsection{研究目标}

\subsection{主要研究内容及拟解决的相关科学问题和技术问题}
	
\subsection{拟采取的研究方法、技术路线、实施方案及可行性分析}

\subsection{预期创新点}

很多内容……
\\[10 cm]
\end{mdframed}


%%% ++++++++++++++++++++++++++++++++++++++++++++++++++++++++++++++++++++++++++++++++++
%
\clearpage
%%%%% --------------------------------------------------------------------------------
%%
%%%%******************************* Main Content *************************************
%%
%%% ++++++++++++++++++++++++++++++++++++++++++++++++++++++++++++++++++++++++++++++++++




\section{研究条件}
\begin{mdframed}[everyline=true]

{\bfseries \kaishu \zihao{5} 开展研究应具备的条件及已具备的条件,可能遇到的困难与问题和解决措施。}
\\[20 cm]
\end{mdframed}

%%% ++++++++++++++++++++++++++++++++++++++++++++++++++++++++++++++++++++++++++++++++++
%
\clearpage
%%%%% --------------------------------------------------------------------------------
%%
%%%%******************************* Main Content *************************************
%%
%%% ++++++++++++++++++++++++++++++++++++++++++++++++++++++++++++++++++++++++++++++++++




\section{学位论文工作计划}
{
\noindent
\begin{tabular*}{0.999\textwidth}{| p{0.07\textwidth } <{\centering} | p{0.45\textwidth}  | p{0.162\textwidth} | p{0.20\textwidth}  |}

	\hline 
	序号 & 	\multicolumn{1}{c}{主要研究内容} & 	\multicolumn{1}{|c}{起讫日期} & 	\multicolumn{1}{|c|}{预期成果} \\
	\hline 
	1  &  内容 &  \tabincell{c}{2017年01月 \\至 \\2017年12月} & 很多论文 \\ 
	\hline 
	2   &  &  &  \\ 
	\hline 
	3    &  &  &  \\ 
	\hline 
	4    &  &  &  \\ 
	\hline 
	5    &  &  &  \\ 
	\hline 
\end{tabular*} 
\indent
}


%%% ++++++++++++++++++++++++++++++++++++++++++++++++++++++++++++++++++++++++++++++++++
%
\clearpage

%%%%% --------------------------------------------------------------------------------
%%
%%%%******************************* Main Content *************************************
%%
%%% ++++++++++++++++++++++++++++++++++++++++++++++++++++++++++++++++++++++++++++++++++


\section{主要参考文献}
\printbib
%\printbibtabular[title={~}]
%\printbibliography[title={~}]

%%% ++++++++++++++++++++++++++++++++++++++++++++++++++++++++++++++++++++++++++++++++++
%
\clearpage
%%%%% --------------------------------------------------------------------------------
%%
%%%%******************************* Main Content *************************************
%%
%%% ++++++++++++++++++++++++++++++++++++++++++++++++++++++++++++++++++++++++++++++++++




\section{指导教师对开题报告的评语}
\begin{mdframed}[everyline=true]
   \indent
   
   论文选题源自应用中的实际需求,针对监控视频中的异常事件检测方法开展研究,对 视频智能监控领域和在视频监控实际应用都具有重要意义。
   
   报告对监控视频异常检测方法的国内现状种进行了综述,指出目前有两类主要方法: 对异常视频序列建模的方法和对正常视频序列建模的方法,并对后者更为具体的方法、模 型进行了详细阐述。参阅的论文范围合理,代表性强,为论文研究进一步开展并提出新方 法奠定了基础。
   
   报告提出的两个研究内容针对监控视频异常事件监测中的两个关键环节,拟采取的研 究方法合理、技术路线清晰,初步试验结果表明报告提出方法的可行性。
论文开展研究需要的基础条件具备,对可能遇到的困难与问题有清楚的认识并有对应 的解决措施。论文研究计划合理,参考文献格式规范。
\\[8 cm]
\end{mdframed}

%%% ++++++++++++++++++++++++++++++++++++++++++++++++++++++++++++++++++++++++++++++++++
%
\clearpage
%%%%% --------------------------------------------------------------------------------
%%
%%%%******************************* Main Content *************************************
%%
%%% ++++++++++++++++++++++++++++++++++++++++++++++++++++++++++++++++++++++++++++++++++


\section{开题报告评议小组意见及评议结果}
\begin{mdframed}[everyline=true]

\begin{enumerate}[label={(\arabic*)},labelsep= 3 pt]
	\item {\songti 选题依据、研究内容、研究方案及技术路线的科学性、可行性及创新性的评价}
	
	\quad\quad  谭XX同学的硕士学位论文深入分析国内外的理论观点和技术方案,对监控视频异常事 件检测思路比较清晰,研究方法具有创新性,选题具有重要的应用价值。论文研究内容与工 作量适合硕士学位论文的要求,论文研究方法可行。
	经评议小组讨论,一致同意谭XX同学的硕士学位论文开题报告。
	
	\item  {\songti 存在的主要问题和修改建议 }
	\begin{enumerate}[label={\arabic*)},labelsep=3 pt]
		\item 论文不必提出“广义异常事件”的概念,只需说明工作内容为异常事件检测而不是区分 异常事件的具体类别。
		\item 异常事件检测如果不限定应用场景范围难度和工作量较大,建议先设定一个具体的场景 以降低难度,逐步推进。
		\item 光流的导数(加速度)也可以作为特征向量的一个维度。
		\item 如果采用的方法是无监督的可能更具有实用性。
	\end{enumerate}

	\item  {\songti 开题报告评议结果}
	
    $\square$  {\songti 通过} \quad\quad \quad \quad  $\square$ {\songti 不通过,且要求在2个月内重新组织开题}
	\\[20pt]
	
	
	{  \songti
	\quad\quad\quad\quad \quad\quad\quad\quad 	\quad\quad\quad\quad \quad\quad\quad\quad 组长(签名):
	
	\quad\quad\quad\quad \quad\quad\quad\quad 	\quad\quad\quad\quad \quad\quad\quad\quad \quad\quad\quad 年  \quad\quad 月  \quad\quad 日
	\\
    }
\end{enumerate}
\end{mdframed}
\vspace{-4pt}
{
	\noindent
\begin{tabular*}{0.999\textwidth}{| c  | c | c | p{0.5\textwidth}< {\centering} | p{0.157\textwidth}<{\centering}|}
%	\hline 
    \multicolumn{5}{|c|}{	\songti 开题报告评议小组组成}	\\
	\hline
	{\songti 组成} & {\songti 姓名} & {\songti 职称} &  {\songti  所在单位} & {\songti 本人签名}  \\
	\hline 
	{\songti 组长}     &  张老师 & 教授 &  五院XXXX研究所 &  \\ 
	\hline 
	\multirow{4}{8pt}{\songti 成员}  & 李老师  &  教授&  五院XXXX研究所 & \\ 
	\cline{2-5}
	   &  王老师 &  教授&  X院XXXX研究所 & \\ 
	\cline {2-5}
	   &  谭老师 &  教授&  五院XXXX研究所 & \\ 
	\cline {2-5}
	   &  老老师 &  教授&  五院XXXXXX系 & \\ 
	\hline 
	{\songti 秘书}   & 赖老师 & 讲师 &  五院XXXXXX系 & \\ 
	\hline 
\end{tabular*} 
   \indent
}


%%% ++++++++++++++++++++++++++++++++++++++++++++++++++++++++++++++++++++++++++++++++++
%
\clearpage

\end{document}
%%%%% --------------------------------------------------------------------------------
